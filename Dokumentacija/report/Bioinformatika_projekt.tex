\documentclass[times, utf8, seminar, numeric]{fer}
\usepackage{booktabs}
 \usepackage{url}

\begin{document}

% Ukljuci literaturu u seminar
\nocite{*}

\title{Poboljšanje djelomično sastavljenog genoma dugim očitanjima}

\author{Bruno Kovač, Tonko Sabolčec, Fabijan Čorak}

\voditelj{doc. dr. sc. Krešimir Križanović}

\maketitle

\tableofcontents

\chapter{Uvod}
Sekvenciranje genoma svodi se na kombiniranje očitanja u jednu cjelinu. Ovaj rad pretpostavlja da su očitanja već sastavljena, ali djelomično - u fragmente. Jedan takav fragment naziva se \textit{contig}. Dakle, zadatak se svodi na što bolje povezivanje \textit{contiga}, što smo učinili oslanjanjem na duga očitanja, kao što je predloženo u \cite{Du345983}.

\chapter{Postupak}
\section{Izgradnja grafa}
\section{Vizualizacija}
\section{Obilazak grafa}

\chapter{Rezultati}
Implementacija je ispitana na tom tom i tom

\begin{center}
\begin{tabular}{|c|c|c|}
	\hline
	E. Coli & X & Y \\
	
	\hline
\end{tabular}
\end{center}
\chapter{Zaključak}
Zaključak.

\bibliography{literatura}
\bibliographystyle{fer}

\chapter{Sažetak}
Sažetak.

\end{document}
